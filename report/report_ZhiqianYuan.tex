\documentclass{report}
\usepackage{graphicx}
\begin{document}
\title{GUI for mail-merge using Erlang and Erlguten}
\author{Yuan Zhiqian}
\date{April 2011}
\maketitle
\begin{abstract}
  Klarna AB is a financial company which provides payment solutions for the e-commerce sector, and they need to generate a lot of invoices everyday, which involves a large amount of tedious work, so they need a tool to facilitate this task. The new method of invoice generating consists of two sub-process, i.e. designing templates and make mail-merging between templates and userdata。

  This project aims for implementing a system to solve the issues described above, it mainly targets on the mail-merging of templates and userdata, besides it also provides a prototype of GUI for designing the templates. The final goal is to realize the automation of invoice generating to the greatest extent, later in this report I will illustrate how far this goal is reached.

  This system involves many technique areas, the techniques mainly used are Erlang, Erlguten, Cappuccino framework, javascript, rubyonrails etc.
\end{abstract}

\tableofcontents

\chapter{Introduction}
  This chapter introducts the project by generally describing it and giving its background. At the end of this chapter, the report structure is shown.
\section{Project Overview}
  
\section{Background}
\section{Report structure}

\chapter{Problem Description}
\section{Issues}
\section{Goals and motivation}

\chapter{Implementation}
\section{System Evironment}
\section{Techniques Involved}
\section{System Structure}
\section{Implementation for each part}
\subsection{Template Format}
\subsection{Mail-merging algorithm}
\subsection{GUI Implementation}
\subsubsection{Template Generating}
\subsection{Combination of GUI and backend system}

\chapter{Evaluation}
\section{Mail-merging}
\subsection{Text}
\subsection{Image}
\subsection{Table}
\subsection{List}
\section{GUI}

\chapter{Future work}
\section{Mail-merging part}
\section{GUI part}

\chapter{Summary and Conclusions}


\end{document}
